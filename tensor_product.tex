\documentclass[10pt]{tufte-handout}

% Packages
\usepackage{amsmath, amssymb, amsthm, mathtools}
\usepackage{tikz-cd} % Back to original for main diagrams
\usepackage[all,cmtip]{xy} % For Leinster-style diagram only
\usepackage{dsfont} % for double-struck letters like \mathds{k}

\renewcommand{\baselinestretch}{1.0}

% Title
\title{Tensor Products and Universal Properties}
\author{Bertold Sedlak}
\date{\today}

% Document
\begin{document}
\maketitle

\begin{center}
\large Summary and Reflections on Exercise 5.1\\
\textit{Introduction to Category Theory – University of Vienna, ST 2025}
\end{center}
\vspace{1em}

\begin{abstract}
This is my (presentation) summary of Exercise 5.1 on Tensor Products.
\end{abstract}

\section*{Exercise 5.1 Statement}
Let $\mathds{k}$ be a field and let $V, W \in \texttt{Vect}_{\mathds{k}}$. The tensor product of $V$ and $W$ is a $\mathds{k}$-vector space $V \otimes_{\mathds{k}} W$ together with a bilinear map $\tau : V \times W \to V \otimes_{\mathds{k}} W$ satisfying the following universal property: for every $U \in \texttt{Vect}_{\mathds{k}}$ and every bilinear map $b : V \times W \to U$, there exists a unique linear map $t : V \otimes_{\mathds{k}} W \to U$ such that the diagram

\marginnote{We use the dashed line to represent a unique morphism}
\begin{center}
\begin{tikzcd}
V \times W \arrow[r, "\tau"] \arrow[dr, swap, "b"] & V \otimes_{\mathds{k}} W \arrow[d, dashed, "\exists!\, t"] \\
& U
\end{tikzcd}
\end{center}

commutes. We write $v \otimes w := \tau(v, w)$ for all $v \in V$ and $w \in W$, and refer to $V \otimes_{\mathds{k}} W$ as the tensor product of $V$ and $W$.

\section{(i) Uniqueness up to Unique Isomorphism}
Show that if $V \otimes_{\mathds{k}} W$ exists, then it is unique up to unique isomorphism.
\marginnote{We only use the universal property, so this applies to any object defined by such a property.}

Let us assume we have two 'tensor' (or universal) spaces $(T, \tau)$ and $(T', \tau')$, both satisfying the universal property with respect to $V \times W$. Then for the bilinear map $\tau' : V \times W \to T'$, the universal property of $(T, \tau)$ gives a unique linear map $t : T \to T'$ such that $t \circ \tau = \tau'$. Similarly, the universal property of $(T', \tau')$ applied to $\tau$ yields a unique linear map $t' : T' \to T$ such that $t' \circ \tau' = \tau$.

We can represent this situation with the following diagram:
\marginnote{This is essentially Leinster's diagram from Chapter 0 (see p.5).}
\[
\xymatrix{
& T \ar[d]^{t} \\
V \times W \ar[ur]^{\tau} \ar[r]|-{\tau'} \ar[dr]_{\tau} & T' \ar[d]^{t'} \\
& T
}
\]



Now observe:
\marginnote{The same argument applied symmetrically to $T'$ gives $t \circ t' = \mathrm{id}_{T'}$.}
\begin{align*}
    t' \circ t \circ \tau &= t' \circ \tau' = \tau, \\
    \Rightarrow\quad t' \circ t &= \mathrm{id}_T \quad \text{by uniqueness.} 
\end{align*}

This shows that $t$ and $t'$ are inverse isomorphisms, so $T \cong T'$.


Therefore, $t$ is an isomorphism between $T$ and $T'$, proving uniqueness up to unique isomorphism.

\section{(ii) Construction Using Bases}
Let $B$ and $C$ be bases of $V$ and $W$. Consider the free vector space $\mathds{k}\langle B \times C \rangle$ and define:
\[ \tau(b, c) = \delta_{(b, c)} \]
where $\delta_{(b,c)}$ denotes the formal generator for the pair $(b, c) \in B \times C$.

We define this space more precisely as:
\[
\mathds{k}\langle B \times C \rangle := \left\{ \sum_{(b, c) \in F} \lambda_{b,c} \cdot (b, c) \;\middle|\; F \subseteq B \times C \text{ finite},\ \lambda_{b,c} \in \mathds{k} \right\}
\]

We extend $\tau$ bilinearly. Given $(v,w) \in V \times W$ with
\[
   v = \sum_i \lambda_i b_i, \quad w = \sum_j \mu_j c_j,
\]
we define:
\[ \tau(v,w) := \sum_{i,j} \lambda_i \mu_j \cdot \delta_{(b_i, c_j)}. \]
This ensures bilinearity because:
\begin{align*}
   \tau(rv + sv', w) &= r\tau(v,w) + s\tau(v',w), \\
   \tau(v, rw + sw') &= r\tau(v,w) + s\tau(v,w')
\end{align*}
for all $r,s \in \mathds{k}$ and $v,v' \in V$, $w,w' \in W$.

\paragraph{Universal Property.} Let $b : V \times W \to U$ be any bilinear map. Define a linear map $t : \mathds{k}\langle B \times C \rangle \to U$ by:
\[ t(\delta_{(b,c)}) := b(b,c). \]
Then for any $(v,w)$ as above,
\marginnote{We do the double sum in one step for brevity.}
\[
 b(v,w) = \sum_{i,j} \lambda_i \mu_j \cdot b(b_i, c_j) = \sum_{i,j} \lambda_i \mu_j \cdot t(\delta_{(b_i, c_j)}) = t(\tau(v,w)).
\]
Thus $t \circ \tau = b$, and $t$ is linear by construction. Uniqueness follows from the fact that $t$ is defined on the basis elements $\delta_{(b,c)}$.



\section{(iii) Base-Free Construction}
Define $\widetilde{\tau}(v, w) := \delta_{(v,w)}$ into the free vector space $\mathds{k}\langle V \times W \rangle$, and define $R$ as the subspace generated by the following relations:
\begin{align*}
\delta(v + v', w) - \delta(v, w) - \delta(v', w), \\
\delta(v, w + w') - \delta(v, w) - \delta(v, w'), \\
\delta(\lambda v, w) - \lambda\delta(v, w), \\
\delta(v, \lambda w) - \lambda\delta(v, w)
\end{align*}
\marginnote{We define $R$ (kernel/nullspace) to enforce bilinear identities.
E.g., $\delta(v+v', w) - \delta(v, w) - \delta(v', w) \in R$ implies we identify $\delta(v+v', w) \sim \delta(v,w) + \delta(v', w)$.
as formally there is no reason to identify them!
}

\noindent
Then define $T := \mathds{k}\langle V \times W \rangle / R$ show that $\widetilde{\tau}: V \times W \to \mathds{k} \langle V \times W \rangle / R$ satisfies the universal property.


\marginnote{\textbf{Remark:} The free vector space $\mathds{k}\langle V \times W \rangle$ consists of \emph{all} finite linear combinations of pairs $(v,w)$ with $v \in V$ and $w \in W$. These are not linearly independent in general -- e.g., 
\[ \lambda_1 (1,0) + \lambda_2 (0,1) + \lambda_3 (1,1) \]
may exhibit linear dependence depending on the imposed relations. Thus, $R$ is necessary to reflect bilinearity.}


\paragraph{Universal Property.} Let $b : V \times W \to U$ be any bilinear map. Define $t : T \to U$ by:
\[ t(\delta(v,w) + R) := b(v,w). \]
This is well-defined because $b$ is bilinear and $R$ captures exactly the relations needed for bilinearity. 

We now have:
\[
   t(\tau(v,w)) = t(\pi(\delta(v,w))) = b(v,w),
\]
so $t \circ \widetilde{\tau} = b$ as desired.

Uniqueness follows because $\widetilde{\tau}$ maps all of $V \times W$ (not just a basis), and any linear $t'$ such that $t' \circ \widetilde{\tau} = b$ must coincide with $t$ on all elements, hence be equal.


\section{(iv) Computational Rules}
We now explain and justify the familiar tensor identities:
\begin{align*}
(v + v') \otimes w &= v \otimes w + v' \otimes w, \\
v \otimes (w + w') &= v \otimes w + v \otimes w', \\
(\lambda v) \otimes w &= \lambda(v \otimes w) = v \otimes (\lambda w)
\end{align*}
These follow directly from the relations used to define the quotient space $T := \mathds{k}\langle V \times W \rangle / R$, and thus hold by construction.

They express bilinearity of the map $\tau : V \times W \to T$ in a concise in-line notation:
\begin{align*}
\tau(v + v', w) &= \tau(v, w) + \tau(v', w), \\
\tau(v, w + w') &= \tau(v, w) + \tau(v, w'), \\
\tau(\lambda v, w) &= \lambda \tau(v, w) = \tau(v, \lambda w).
\end{align*}

\marginnote{This is reminiscent of multiplication:
\[ m(a, b) := a \cdot b \]}

This bilinearity is exactly what ensures that any bilinear map $b : V \times W \to U$ factors uniquely through $\tau$, via a linear map $t : V \otimes W \to U$.

\marginnote{These rules are defining properties of the tensor symbol $\otimes$.}


\section{(v) The Tensor Product as a Functor}
Lift the operation $(-) \otimes_{\mathds{k}} (-)$ to a functor.
\marginnote{We now use $T$ for the functor, not the tensor space itself.}
\[
T : \texttt{Vect}_{\mathds{k}} \times \texttt{Vect}_{\mathds{k}} \to \texttt{Vect}_{\mathds{k}}
\]

The essential structure is already in place:

\begin{itemize}
  \item \textbf{On objects:} for $(V, W) \in \texttt{Vect}_{\mathds{k}} \times \texttt{Vect}_{\mathds{k}}$, define $T(V, W) := V \otimes W$.

  \item \textbf{On morphisms:} a morphism $(\varphi, \psi): (V, W) \to (V', W')$ consists of linear maps $\varphi: V \to V'$, $\psi: W \to W'$. Define:
  \[
  T(\varphi, \psi)(v \otimes w) := \varphi(v) \otimes \psi(w)
  \]
  This is well-defined and respects bilinearity and linearity, hence it is a morphism in $\texttt{Vect}_{\mathds{k}}$.
\end{itemize}

\marginnote{We omit $\otimes_\mathds{k}$ subscripts when the field is fixed contextually.}

Identity and composition are preserved:
\begin{align*}
T(\text{id}_V, \text{id}_W)(v \otimes w) &= v \otimes w, \\
T(\varphi_2, \psi_2) \circ T(\varphi_1, \psi_1)(v \otimes w) &= (\varphi_2 \circ \varphi_1)(v) \otimes (\psi_2 \circ \psi_1)(w).
\end{align*}

Thus $T$ is indeed a functor.

\marginnote{Tensoring commutes with morphisms: the diagram illustrates how morphisms are "distributed" by $T$.}

\begin{center}
\begin{tikzcd}[column sep=huge, row sep=huge]
(V, W) \arrow[d, mapsto, "{(\varphi, \psi)}"'] \arrow[r, mapsto] & V \otimes W \arrow[d, mapsto, "\varphi \otimes \psi"] \\
(V', W') \arrow[r, mapsto] & V' \otimes W'
\end{tikzcd}
\end{center}

\section*{References}
This summary is based on the material and exercises from:
\begin{itemize}
    \item Tom Leinster, \textit{Basic Category Theory} (Cambridge University Press)
    \item Steven Roman, \textit{Advanced Linear Algebra} (Springer)
\end{itemize}

\end{document}
